% Tento soubor nahraďte vlastním souborem s přílohami (nadpisy níže jsou pouze pro příklad)
% This file should be replaced with your file with an appendices (headings below are examples only)

% Umístění obsahu paměťového média do příloh je vhodné konzultovat s vedoucím
% Placing of table of contents of the memory media here should be consulted with a supervisor
%\chapter{Obsah přiloženého paměťového média}

%\chapter{Manuál}

%\chapter{Konfigurační soubor} % Configuration file

%\chapter{RelaxNG Schéma konfiguračního souboru} % Scheme of RelaxNG configuration file

%\chapter{Plakát} % poster

\chapter{Riešenie rôznych typov diferenciálnych rovníc}

\subsection{Riešenie diferenciálnej rovnice s~operáciou násobenia} \label{priloha_nasobenie}
Předpokládejme zadanou obecnou diferenciální rovnici se zadanou počáteční podmínkou:
\begin{equation}
\label{nasobeni_integrace}
y'= q \cdot r~,~~~~~~y(0)=y_0
\end{equation}

Pro každou proměnnou v~rovnici je potřeba vytvořit Taylorovu řadu. Taylorovy řady pro zadanou rovnici vypadají následovně:
\begin{eqnarray}
y_{i+1} = DY0_i + DY1_i + DY2_i + DY3_i + \cdots + DY(n)_i\\
q_{i+1} = DQ0_i + DQ1_i + DQ2_i + DQ3_i + \cdots + QZ(n)_i\\
r_{i+1} = DR0_i + DR1_i + DR2_i + DR3_i + \cdots + DR(n)_i
\end{eqnarray}

Výpočet Taylorovy řady pro rovnici součinu se počítá následujícím způsobem:\\\\
Nultý člen Taylorovy řady je roven počáteční podmínce:
\begin{equation}
DY0 = y_0
\end{equation}

Vyjádření prvního členu Taylorovy řady je následující:
\begin{equation}
DY1 = h \cdot y' 
\end{equation}
Podle vztahu (\ref{nasobeni_integrace}) je $y'$ na hrazen vztahem $q \cdot r$:
\begin{equation}
\label{h*q*r}
DY1 = h \cdot q \cdot r
\end{equation}


Zderivujeme počáteční funkci (\ref{nasobeni_integrace}) a dostaneme vztah: 
\begin{equation}
y'' = q' \cdot r + q \cdot r'
\end{equation}

Nahradíme derivace pomocí prvků Taylorovy řady.
\begin{equation}
DY2 = \frac{h^2}{2!} \cdot ( \frac{DQ1}{h} \cdot r + q \cdot \frac{DR1}{h})
\end{equation}

Obdobně postupujeme u~třetího prvku další derivací funkce:
\begin{equation}
y''' = q'' \cdot r + 2 \cdot q' \cdot r' + q \cdot r''
\end{equation}

Opět nahradíme derivace členy Taylorovy řady a vyjádříme třetí člen:
\begin{equation}
DY3 = \frac{h^3}{3!} (\frac{DQ2}{\frac{h^2}{2!}} \cdot r + 2 \cdot \frac{DQ1}{h} \cdot \frac{DR1}{h} + q \cdot \frac{DR2}{\frac{h^2}{2!}})
\end{equation}
\\
Obdobně postupujeme při vyjadřování dalších členů.
Jednotlivé členy lze následně upravit pokrácením. Zde jsou uvedeny první čtyři členy:
\begin{eqnarray}
DY1 &=& h \cdot q \cdot r\\
DY2 &=& \frac{h}{2} \cdot ( DQ1 \cdot DR0 + DQ0 \cdot DR1)\\
DY3 &=& \frac{h}{3} ( DQ2 \cdot DR0 + DQ1 \cdot DR1 + DQ0 \cdot DR2)\\
DY4 &=& \frac{h}{4} (DQ3 \cdot DR0 + DQ2 \cdot DR1 + DQ1 \cdot DR2 + DQ0 \cdot DR3)
\end{eqnarray}

Z~těchto vztahů je patrné, že jednotlivé členy se rozvíjejí a tento rozvoj lze algoritmizovat.\\ 
%Výsledný vztah pro výpočet i-tého členu vypadá následovně
%
%\begin{equation}
%\begin{split}
%\label{rovnice_clenu_Tay_nasobeni}
%DY_{i} = \frac{h}{i} (DQ_{i-1} \cdot DR_{i-i} + DQ_{i-2} \cdot DR_{i-(i-1)} +  \\  + DQ_{i-(i-1)}\cdot DQ_{i-2} + DQ_{i-i} \cdot DR_{i-1})
%\end{split}
%\end{equation}


\subsection{Riešenie diferenciálnej rovnice s~operáciou delenia} \label{priloha_delenie}

Diferenciálna rovnica s~operáciou delenia:
\begin{eqnarray}
y' & = & \dfrac{u}{v}
\end{eqnarray}

Ďalšie derivácie rovnice \eqref{dif_delenie} sú:
\begin{eqnarray}
y'' = \dfrac{u'v - uv'}{v^{2}} & = & \dfrac{1}{v} (u' - y'v') \label{priloha_derivacie_delenie} \\
y''' = \left( \dfrac{1}{v} (u' - y'v') \right)' & = & \dfrac{1}{v} (u'' - 2y''v' - y'v'') \nonumber \\
y^{(4)} = \left( \dfrac{1}{v} (u'' - 2y''v' - y'v') \right)' & = & \dfrac{1}{v} (u''' - 3y'''v' - 3y''v'' - y'v''') \nonumber \\
 & \vdots \nonumber &
\end{eqnarray}


Podobne ako členy \ref{DY_cleny}, aj členy $ DV(n)_{i} $ a členy $ DU(n)_{i} $:
\begin{align}
DV1_{i} &= h v_{i}					& DU1_{i} &= h u_{i} \ \\ 
DV2_{i} &= \dfrac{h}{2} DV1_{i} 	& DU2_{i} &= \dfrac{h}{2} DU1_{i} \nonumber \\
DV3_{i} &= \dfrac{h}{3} DV2_{i} 	& DU3_{i} &= \dfrac{h}{3} DU2_{i} \nonumber \\
DV4_{i} &= \dfrac{h}{4} DV3_{i} 	& DU4_{i} &= \dfrac{h}{4} DU3_{i} \nonumber \\
&\hspace{2mm}\vdots					& &\hspace{2mm}\vdots \nonumber \\
DV(N)_{i} &= \dfrac{h}{n} DV(N-1)_{i} & DU(N)_{i} &= \dfrac{h}{n} DU(N-1)_{i} \nonumber
\end{align}

Po dosadení jednotlivých členov $ DV(n)_{i} $ a $ DU(n)_{i} $ do derivácií \ref{priloha_derivacie_delenie} dostaneme:
\begin{eqnarray}
\dfrac{DY1_{i}}{h} & = & \frac{1}{v} u~\\
\dfrac{DY2_{i}}{\frac{h^{2}}{2!}} & = & \dfrac{1}{v} ( \dfrac{DU1_{i}}{h} - \dfrac{DY1_{i}}{h}\dfrac{DV1_{i}}{h} ) \nonumber \\
\dfrac{DY3_{i}}{\frac{h^{3}}{3!}} & = & \dfrac{1}{v} 
( \dfrac{DU2_{i}}{\frac{h^{2}}{2!}} - 
2\dfrac{DY2_{i}}{\frac{h^{2}}{2!}} \dfrac{DV1_{i}}{h} - 
\dfrac{DY1_{i}}{h} \dfrac{DV2_{i}}{\frac{h^{2}}{2!}} ) \nonumber \\
\dfrac{DY4_{i}}{\frac{h^{4}}{4!}} & = & \dfrac{1}{v} 
( \dfrac{DU3_{i}}{\frac{h^{3}}{3!}} - 
3\dfrac{DY3_{i}}{\frac{h^{3}}{3!}} \dfrac{DV1_{i}}{h} - 
3\dfrac{DY2_{i}}{\frac{h^{2}}{2!}} \dfrac{DV2_{i}}{\frac{h^{2}}{2!}} -
\dfrac{DY1_{i}}{h} \dfrac{DV3_{i}}{\frac{h^{3}}{3!}} ) \nonumber \\
& \vdots \nonumber & 
\end{eqnarray}

Po úprave vyzerajú jednotlivé členy Taylorovej rady \eqref{Taylor} pre riešenie diferenciálnej rovnice \eqref{dif_delenie} nasledovne:
\begin{eqnarray}
DY1_{i} & = & \frac{1}{v} (hu)  \\
DY2_{i} & = & \dfrac{1}{2v} (DU1_{i}h - DY1_{i}DV1_{i}) \\
DY3_{i} & = & \dfrac{1}{3v} ( DU2_{i}h - 2DY2_{i}DV1_{i} - DY1_{i}DV2_{i} ) \\
DY4_{i} & = & \dfrac{1}{4v} ( DU3_{i}h - 3DY3_{i}DV1_{i} - 2DY2_{i}DV2_{i} - DY1_{i}DV3_{i} ) \\ 
& \vdots \nonumber & 
\end{eqnarray}


Formuly jednotlivých členov tvoria Pascalov trojuholník a sú základom pre tvorbu návrhu deliaceho integrátora.
